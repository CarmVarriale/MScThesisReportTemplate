\chapter{Research Proposal}
\label{ch:research-proposal}
%
\epigraph{
}{}


% The purpose of the Research Proposal is to guide yourself forward in your research project. 
%
% The Research Proposal takes its most effective form when it is formulated in terms of one Research Question.
%
% The Research Question should be your guiding light for the entire project.
% You will have to formulate it so that it can be used throughout the entirety of the research project to keep the right direction and avoid getting sidetracked.
%
% The Research Question is well formulated if they respect the S.M.A.R.T. paradigm. It is particularly effective if they are Specific and Measurable. 
%
% To evaluate if a Research Question is well formulated, try asking yourself the following questions:
% 1. What is the message I want to communicate at the end of the project?
% 2. Can I answer this question with a single chart, or a couple of charts?
%   - What variables would be plotted on the axes?
%   - How would the caption describe the chart?
%   - What shape do I expect on the chart? = Research Hypothesis
%
% The Research Question can change throughout the project, and most likely will. This is because your understanding of the state-of-the-art (where you are) and proposed methodology (how you want to move forward) are going to keep evolving through the project itself. So, it makes sense to periodically re-evaluate if the Research Questions still serve its guiding purpose in light of the new developments.
%
% The Research Question must be formulated explicitly in the Research Proposal, and it should be explicitly answered by the end of it.
%
% In order to stay on point, plan your Research Proposal to be in the order of at most 2 pages.