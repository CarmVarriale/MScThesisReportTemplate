\chapter{State-of-the-art review}
\label{ch:soa-review}
%
\epigraph{
}{}

% The State-of-the-Art Review should present a critical evaluation of what has been done in the past to address the problem of interest (or similar problems).
%
% It can cover theoretical aspects, experimental aspects, numerical aspects, or a combination of these.
% It can be structured in a chronological way, or in a thematic way, or in a combination of these.
%
% It can cover fundamental bases as well as practical engineering products, as long as they are relevant to the specific problem of interest.
% If something is extremely relevant, then it is probably worth reviewing in detail. If it is only marginally relevant, then it can only just be mentioned in passing, and referenced in the bibliography.
%
% The reader should not spend time reading something that will not be useful in the rest of the report.
%
% The review should be critical, meaning that you should not just summarize what has been done in the past, but also discuss the strengths and weaknesses of the different approaches.
%
% It should culminate in a discussion of why this was not enough to solve the specific problem of interest, highlighting the so-called "knowledge gap".
%
% You may find it useful to shape the State-of-the-Art Review around the following questions:
%
% 1. What did we do in the past to solve this problem? = Review
% 2. What was good and what was bad about those attempts? = Critical evaluation
% 3. In what form is the problem now? = Knowledge gap
%
% In order to stay on point, plan your State-of-the-Art Review to be in the order of a 15-25 pages.