\chapter{MSc Thesis Process}
\label{ch:msc-thesis-process}
%
\epigraph{
	Teams move at the speed of trust. \\
    Be the kind of dependable person \\
    you would want to work with.
}{Joel Goldberg}


The MSc Thesis project is a learning experience: the last learning experience of your career as a university student. 
Since it is a learning experience, the MSc Thesis project is a course with its own course code and its official entry in the BrightSpace Catalog.
These are the official reference pages for the MSc Thesis project at the Aerospace Engineering Faculty, for the MSc Track \quotes{Flight Performance and Propulsion}:
\begin{itemize}
    \item Master Thesis Aerospace Engineering: \url{https://brightspace.tudelft.nl/d2l/home/623070}
    \item Study guide: \url{https://www.studiegids.tudelft.nl/courses/study-guide/educations/20759}
\end{itemize}




\section{Learning objectives}

By the end of the MSc Thesis project, you should be able to design and manage a research project independently, from the beginning (definition) to the end (dissemination). 
In order to develop those skills, \emph{you} are going to be put in charge of moving the research project forward in all of its aspects (administration, logistics, scientific content, ...).


In the Brightspace page, you can find the \emph{Rubric} used by the committee for the final assessment.
Please use this rubric to understand how your efforts are going to be evaluated in the end, and guide your conduct accordingly.
%
As you can see from the rubric, you are not only going to be evaluated on the basis of your scientific knowledge and your capability to develop an engineering methodology, but also on your ability to make a plan (and update it accordingly), formulate useful research questions, communicate effectively both orally and in writing, and manage everyone's time and resources efficiently.
%
The supervisory team is responsible to:
\begin{itemize}
    \item act as a sounding board for the ideas that \emph{you} propose;
    \item advise you or constructively oppose you on your hypotheses, methodologies, and conclusions;
    \item support you in taking the most appropriate decisions;
    \item facilitate or enable your progress, and remove obstacles (if any);
    \item navigate you through the procedures by clarifying expectations.
\end{itemize}


\section{Milestones}

The MSc Thesis project is structured around four main milestones: research proposal review meeting (flexible), mid-term review meeting, green-light review meeting, and final defence.
You can see what is expected from you at each milestone on \href{https://brightspace.tudelft.nl/d2l/home/623070}{Brightspace}.
The milestones are not just deadlines to be met, but opportunities to get feedback on your progress and re-evaluate your plan accordingly.
You should organize your work in a way to get \emph{the most valuable} feedback at each milestone, in a way that \emph{you can act} upon it in a timely manner, in order to reach your goals.


At each milestone, you will have to deliver a presentation and the latest version of the report.
These are also the deliverables that you will be evaluated upon at the end of the project, so it is a good idea to work on them in that perspective already.
You will get feedback on the scientific and engineering approach, on your communication skills, on your presentation and on the latest additions to the report (primarily but not exclusively).
In this way, you will be able to build up on the feedback directly to improve the quality of your final work, the one that will actually earn your grade.


The version of the report that is expected at each milestone is indicated in the form of comments in the \texttt{main.tex} file and other files in the main body of the document.
The presentation should last about 30 minutes.
It does not have a template, so feel free to give it your sping.
The purpose of the presentation is to communicate your work effectively to a mixed audience, made of peers, friends, family, and experts (committee members).






\section{Scientific paper}
This thesis report requires a scientific article to be part of the deliverable.
This is supposed to be a \emph{completely standalone} document, that can be read and understood without the need to refer to the rest of the report.
%
As it is targeted at an audience of technical experts, it should be written in a concise and precise way, using the conventions of scientific writing.


The scientific article should be written in the format of a journal or conference that is relevant to your research topic, as the best examples you have read yourself through the project.
Consult with your supervisor to choose the most appropriate paper format and template for your specific case.

It is up to you to set up the document folder structure, the \LaTeX template, and the bibliography management system to write the scientific article.
Feel free to reuse parts of this template to your convenience, of course.
Once the scientific paper is complete, export it as a \texttt{.pdf} file named \texttt{sciarticle.pdf}, and put it in the main folder of the report.




\section{Handing over}
At the end of the project, you will have to hand over the material that you have produced during the project.
This includes:
%
\begin{itemize}
    \item the final version of the report as a \texttt{.pdf} file;
    \item the final presentation as a \texttt{.pptx} file;
    \item the \LaTeX source code of the scientific article, in a well organized folder structure;
    \item the code and data that you have produced, in a well organized folder structure, with a \texttt{README.md} file explaining how to use it;
\end{itemize}
%
Do not include libraries or external packages in the code folder structure, but rather provide a list of dependencies in the \texttt{README.md} file.